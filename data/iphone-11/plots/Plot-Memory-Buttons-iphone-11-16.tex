% \documentclass{article}
% \usepackage{pgfplots}
% \usepackage{pgfplotstable} % For reading data
% \usepackage{filecontents}

\pgfplotsset{compat=1.17} % Use the appropriate version of pgfplots

% Read the data.csv file
\pgfplotstableread[col sep=comma]{data/Table-iphone-11-16.csv}\datatable

% Get the first entry from the first column of the data.csv file
\pgfplotstablegetelem{0}{DeviceName}\of\datatable
\edef\plotTitle{\pgfplotsretval}

% \begin{document}

\begin{tikzpicture}
\begin{axis}[
    title=Plot of scenario \textbf{Buttons} on \textbf{\plotTitle}, % Set the title to the table name read from the CSV file
    xlabel=Number of Components,
    ylabel=Memory Usage (kB),
    legend pos=north west,
    ymajorgrids=true,
    ymin=0,
    grid style=dashed,
    xmode=log, % Enable logarithmic scale for x-axis
    log basis x=2, % Set the base of the logarithm to 2
    xtick=data, % Use data from the file for xticks
    xticklabels from table={\datatable}{NumberOfComponents}, % Read xtick labels from the data file
    xticklabel style={rotate=45, anchor=north east}, % Rotate x-axis labels 45 degrees
    yticklabel style={
            /pgf/number format/fixed,
            /pgf/number format/precision=5
    },
    scaled y ticks=false,
    %nodes near coords, % Display plot values near the data points
    %point meta=explicit, % Use explicit point meta data for the values
    %every node near coord/.append style={font=\tiny, yshift=2pt, xshift=2pt}, % Customize the style of the nodes
]

\addplot table [x=NumberOfComponents, y=ButtonsMemory(UIKit), col sep=comma, meta=ButtonsMemory(UIKit)] {\datatable};
\addlegendentry{UIKit}

\addplot table [x=NumberOfComponents, y=ButtonsMemory(SwiftUI), col sep=comma, meta=ButtonsMemory(SwiftUI)] {\datatable};
\addlegendentry{SwiftUI}

\end{axis}
\end{tikzpicture}

% \end{document}
